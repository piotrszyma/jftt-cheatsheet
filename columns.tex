
  \begin{multicols}{3}
    \textbf{DFA}=$(Q,\Sigma , \delta , q_{0},F)$
    \begin{enumerate}
      \item [$Q$]skończony zbiór stanów
      \item [$\Sigma$]skończony alfabet wejściowy 
      \item [$\delta$]funkcja przejścia postaci   $Q\times\Sigma \rightarrow Q$
      \item [$q_{0}$]stan początkowy
      \item [$F \subseteq Q$]zbiór stanów akceptujących
    \end{enumerate}
    \textbf{Minimalizacja DFA}
    \begin{enumerate}
      \item forall p końcowy, q niekońcowy, oznacz (p,q)
      \item forall $(p,q) \in (F\times F) \cup (Q\setminus F \times Q\setminus F ), p \neq q $ if $\exists_{a \in \Sigma} (\delta (p,a), \delta (q,a))$ jest oznaczona, oznacz (p,q) (rekurencyjnie).
      \item nieoznaczone scalamy.
    \end{enumerate}
    \textbf{PDA} $M=(Q,\Sigma , \Gamma , \delta , q_{0} , Z_{0} , F)$
    \begin{enumerate}
      \item[$Q$] skończony zbiór stanów
      \item[$\Sigma$] alfabet wejściowy
      \item[$\Gamma$] alfabet stosowy
      \item[$q_{0} \in Q$] stan początkowy
      \item[$Z_{0} \in \Gamma$] symbol początkowy na stosie
      \item[$F \subset Q$] zbiór stanów akceptujących \\(jeśli $F=\emptyset$ to akceptujemy przez pusty stos)
      \item[$\delta $] funkcja przejścia postaci $\delta: Q \times (\Sigma \cup \lbrace \varepsilon \rbrace)\times \Gamma \rightarrow 2^{Q\times \Gamma ^{*}}$
    \end{enumerate}
    \textbf{LOP } Zał., że $L$ regularny. Wtedy istnieje stała $n$, że jeśli $z \in L$ oraz $|z| \geqslant n$, to można podzielić $z$ na $z=uvw$ takie, że:
    \begin{enumerate}
      \item $|v| \geqslant 1$
      \item $|uv| \leqslant n$
      \item $\forall_{i \in \mathbb{N}} z'=uv^{i}w \in L$
    \end{enumerate}
    Podział $\alpha = uvw$ , $|uv| \leqslant n$ oraz $|v| \geqslant 1$.
    Wybieramy $i$ dla którego $|uv^{i}w| \notin L$ a powinien. \\
    \textbf{LOP bezk.} Zał., że $L$ bezkontekstowy.Wtedy istnieje stała $n$, że jeśli $z \in L$ oraz $|z| \geqslant n$, to można podzielić $z$ na $z=uvwxy,$ takie, że:
    \begin{enumerate}
      \item $|vx| \geqslant 1$
      \item $|vwx| \leqslant n$
      \item $\forall_{i \in \mathbb{N}} z'=uv^{i}wx^{i}y \in L$
    \end{enumerate}
    \textbf{Lemat Ogdena} Niech $L$ język bezkontekstowy. Wtedy istnieje stała $n$ taka, że jeśli $z \in L$ oraz $|z| >= n$ i oznaczymy w $z$ $n$ lub więcej pozycji jako wyróżnione, to można podzielić $z$ na $z = uvwxy$ takie, że:
    \begin{enumerate}
      \item $v$ i $x$ zawierają łącznie co najmniej jedną wyróżnioną pozycję
      \item $vwx$ zawiera co najwyżej $n$ wyróżnionych pozycji
      \item $\forall i \in \mathbb{N} \; z' = uv^{i}wx^{i}y \in L$
    \end{enumerate} 
    \textbf{Klasa języków regularnych} jest domknięta na operację sumy, 
    dopełnienia, przecięcia, złożenia i domknięcia Kleene'ego.
    \textbf{Gramatyka bezkontekstowa} G=(N,T,P,S)
    \begin{enumerate}
      \item[N] - skończony zbiór zmiennych (nieterminale)
      \item[T] - skończony zbiór zmiennych końcowych(terminale, alfabet)
      \item[P] - skończony zbiór produkcji postaci A $\rightarrow \alpha$ gdzie A $\in$ N i $\alpha \in$ (N $\cup$ T)$^{*}$
      \item[S] $\in$ N - \text{symbol początkowy}
    \end{enumerate} 
    \textbf{Postać normalna Chomsky'ego} postaci:\\
    $A \rightarrow BC$ albo $A \rightarrow a$\\
    Konstrukcje:\\
    \begin{enumerate}
      \item If po prawej terminal $a$ to zastępujemy go $C_{a}$ i dopisujemy $C_{a} \rightarrow a$
      \item If prawa strona dłuższa niz 1 to zastępujemy $A \rightarrow B_{1} \ldots B_{n}$ przez \\
    $A \rightarrow B_{1}D_{1}, D_{1} \rightarrow B_{2}D_{2}, \ldots ,D_{n-2} \rightarrow B_{n-1}B_{n}$
    \end{enumerate}
    \textbf{FIRST(X) - dla symboli}
    \begin{enumerate}
      \item X-terminal, to FIRST(X)={X}
      \item X$\rightarrow \varepsilon$ to do FIRST(X) dodajemy $\varepsilon$
      \item X - nieterminal i $X \rightarrow Y_{1}Y_{2}...Y_{k}$ to dodajemy $a$ do $FIRST(X)$ jeśli istnieje $i$ takie, że $a \in FIRST(Y_{i})$ oraz $\varepsilon \in FIRST(Y_{j})$ dla każdego $j<i$. \\ $\varepsilon \in FIRST(X)$ jeśli należy do wszystkich $FIRST(Y_{i})$. 
      \item $FIRST(X\alpha) = FIRST(X)$ gdy $\varepsilon \notin FIRST(X)$
      \item $FIRST(X\alpha) = FIRST(X) \cup FIRST(\alpha)$ gdy $\varepsilon \in FIRST(X)$
    \end{enumerate}
    \textbf{FOLLOW(A) - dla nieterminali}
    \begin{enumerate}
      \item Dla początkowego $S$ do $FOLLOW(S)$ dodajemy \$
      \item Jeśli mamy produkcję $A\rightarrow\alpha B \beta$ to do $FOLLOW(B)$ dodajemy wszystkie symbole z $FIRST(\beta)$ poza $\varepsilon$
      \item Jeśli $A\rightarrow\alpha B $ lub $A\rightarrow\alpha B \beta$, gdzie $\varepsilon \in FIRST(\beta)$ to do $FOLLOW(B)$ dodajemy wszystkie symbole z $FOLLOW(A)$
    \end{enumerate}
    \textbf{LL(1) - $A \rightarrow \alpha$} \\
    Tabela: nazwy kolumn terminale i \$ !!!$ \Updownarrow $ \\ nazwy wierszy nieterminale $ \Leftrightarrow $
    \begin{enumerate}
      \item $\forall$ produkcji $A\rightarrow \alpha$  z gramatyki \\ wykonaj 2 i 3
      \item foreach $a \in T$ if $a \in FIRST(\alpha)$ to wpisz $A\rightarrow \alpha$ do $M[A,a]$
      \item if $\varepsilon \in FIRST(\alpha)$ to dla każdego $b \in FOLLOW(A)$ wpisz $A\rightarrow \alpha$ do $M[A,b]$  Jeżeli $ \varepsilon  \in FIRST(\alpha)$ oraz $ \$ \in FOLLOW(A) $, dodaj $A \rightarrow \alpha $ do $M[A, \$]$ 
      \item \textsc{protip}: nie ma w tabeli $\varepsilon$!
    \end{enumerate}
    \textbf{SLR}
    Tabela:\\ nazwy kolumn \textsc{akcje} (terminale i \$ !!!) \\i \textsc{przejścia} (nieterminale) \\ nazwy wierszy stany 
    
    \begin{enumerate}
      \item zbiory sytuacji $C = I_0, \dots, I_n $ \\ Zaczynamy od $I_0 = domkniecie([S' \rightarrow \centerdot S])$
      \item tabelka + redukcje (zaznaczyć ew. konflikty) \\
            konstrukcja tabelki: w częsci akcji $s_x$ (shift) i $r_x$ (reduce), a w części przejść (nieterminale) $x$ (liczba) \\
            ACC dla $S' \rightarrow S  \centerdot $
      \item redukcja do FOLLOW(A) (if redukcja była z $A\rightarrow \beta \centerdot$)
    \end{enumerate}
    \textbf{LR(1)}
    \begin{enumerate}
      \item zbiory sytuacji z \textsc{podglądem}
      \item podgląd początkowy \$
      \item podgląd przy domknięciu: mamy $[A\rightarrow \alpha \bullet B \beta , a] \in I$ 
          dla każdej produkcji z $B\rightarrow \gamma$ dodaj 
          $[B\rightarrow \centerdot \gamma ,FIRST(\beta a)]$
      \item tabelka jak SLR ale zamiast redukcja do FOLLOW(A) (if redukcja była z $A\rightarrow \beta \centerdot$) to redukcja do elementów z \testsc{podglądu}
    \end{enumerate}	
    \textbf{LALR}
    \begin{enumerate}
      \item generujemy rodzinę $ C = I_0, \dots, I_n $ jak w LR(1)
      \item sklejamy jądra nie patrząc na podgląd, a podglądy łączymy - tabelka analogicznie do LR(1)
    \end{enumerate}	
    \textbf{LEADING(A)-pierwsze term. z A}
    \begin{enumerate}
      \item	$a \in LEADING(A)$ jeśli mamy produkcję $A\rightarrow Ba\beta$ lub $A \rightarrow a \beta$ 
      \item if exists prod. $A\rightarrow B\alpha$ i $a \in LEADING(B)$ to $a \in LEADING(A)$
      \item foreach nieterminali liczymy 1 i powtarzamy 2 aż nic się nie zmienia
    \end{enumerate}
    \textbf{TRAILING(A)-ostatnie term. z A}
    \begin{enumerate}
      \item	$a \in TRAILING(A)$ jeśli mamy produkcję $A\rightarrow \beta aB$ lub $A \rightarrow \beta a$ 
      \item if exists prod. $A\rightarrow \alpha B$ i $a \in TRAILING(B)$ to $a \in TRAILING(A)$
      \item foreach nieterminali liczymy 1 i powtarzamy 2 aż nic się nie zmienia
    \end{enumerate}
    \textbf{Tab. priorytetów $\doteq \lessdot \gtrdot $}
    \begin{enumerate}
      \item [$TT$] $T \doteq T$
      \item [$TNT$] $T \doteq T$
      \item [$TN$] foreach $a \in LEADING(N)$ do $T \lessdot a$ (wiersze) $ \Leftrightarrow $
      \item [$NT$] foreach $a \in TRAILING(N)$ do $a \gtrdot T$ (kolumny) $ \Updownarrow $
      \item [$\$$] zawsze gorszy
    \end{enumerate}
    \textbf{Zbiory sytuacji}
    \begin{enumerate}
      \item Wzbogacenie $ S' \rightarrow S$ 
      \item Ponumerować produkcje (do redukcji!!!).
      \item [$ E \rightarrow \varepsilon $] $E \rightarrow . $
      \item dla kropek, na końcu w tabeli numer z produkcji
    \end{enumerate}
    \textbf{Rekurencja}
    \begin{enumerate}
      \item $A \rightarrow A \alpha | B$
      \item $A \rightarrow \beta A'$
      \item $A' \rightarrow \alpha A' | \varepsilon $
    \end{enumerate}
    \textbf{Faktoryzacja}
    \begin{enumerate}
      \item $A \rightarrow \alpha \beta _{1}|...|\alpha \beta _{k}$
      \item $A \rightarrow \alpha A'$
      \item $A' \rightarrow \beta _{1}|...|\beta _{k}$
    \end{enumerate}
\end{multicols}
 