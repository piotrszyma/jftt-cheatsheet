
\begin{tabular}{l|l|l|l}
  język & lem & slowo & notes\\
  \hline
    $ L = \{ uvv^Rw: u, v, w \in \{0, 1, 2\}^* v \neq \varepsilon \} $ & reg & 
    dwa te same sym obok \\
  \hline
    $ L = \{ wwx: w \in \{0, 1, 2\}^* x \in \{0, 1\}^* |w| > 0 \} $ & Ogd & $21^n \underline{0^n221^n}0^n2$ \\
  \hline
    $ L = \{ wxw^R: w \in \{0, 1, 2\}^* x \in \{0, 1\}^* |w| > 0 \} $ & reg & konczy sie tym czym zaczyna \\
    \hline
    $ L = \{ a^nb^ka^n, n \neq k \} $ & Ogd & dwa warunki to za mało \\
    \hline
    $ L = \{ a^ib^jc^k, i < j < k \} $ & Ogd & nie bezkontekstowy \\
    \hline
    $ L = \{ a^ib^jc^k, k = max(i, j) \} $ & lop/ogd & nie bezkontekstowy \\
    \hline
    $ L = \{ a^ib^jc^k, i + j =k \} $ & lop & bezkontekstowy \\
    \hline
    $ L = \{ a^ib^jc^k, j = i + k \} $ &   & bezkontekstowy \\
    \hline
    $ L = \{ a^nb^mc^n, m \neq n \} $ &   & nie jest bezkontekstowy \\
    \hline
  $ L = \{ ww, w \in \{ 0, 1 \}^* \} $ &   & Nie bezkontekstowe, ale dopełnienie bezkontekstowe \\
  \hline
  $ L = \{ w \in \{ 0, 1, 2 \}^* w = palindrom \; i \; |w|_0 = |w|_2 mod 13\} $ &   & bezkontekstowy, hybryda\\
  \hline
  $ L = \{ w \in \{ a, b, c \}^* w = palindrom \; i \; |w|_a = |w|_b i  |w|_c > 0 \} $ & Ogden  & $ a^nb^ncb^na^n $\\
  \hline
    $\omega = xxy \wedge x \neq \varepsilon $ & LOP & 
    $ab^{n}ab^{n}$ & $i=0$ \\
  \hline 
    $\omega = xyyz \wedge y \neq \varepsilon $ & reg & 
    $len \geqslant 4$ & dobrać krótsze \\
  \hline
    $\omega \omega ^{R} \wedge |\omega|_{a}\equiv |\omega|_{b} \equiv 0 (mod 13) $ & LOP & 
    $a^{13n}b^{13n}b^{13n}a^{13n}$ & ozn. \\
  \hline
  $\omega : |\omega|_{a}\equiv |\omega|_{b}(mod 3) $ & reg & 
  mini & \\
  \hline
    $\omega = xyy^{R} \wedge y \neq \varepsilon$ & reg & 
    2 obok & \\
  \hline
    $\omega:palindrom \wedge |\omega|_{a} = |\omega|_{c} $ & LOP & 
    $a^{n}c^{n}c^{n}a^{n}$ \\
  \hline
    $\omega = xcycz \wedge$ xy i yz $\in \lbrace a,b \rbrace ^{*}$palindromy & Ogd & $a^{m}bca^{m}cba^{m}$ & śr. ozn. \\
  \hline
    $|\omega|_{a} = |\omega|_{b}$ & bezk. & \\
  \hline
    $|\omega|_{a} = |\omega|_{b}= |\omega|_{c}$ & LOP & $a^{n}b^{n}c^{n}$ & \\
  \hline
    $\omega : |\omega|_{a} \neq |\omega|_{b} \neq |\omega|_{c}$ & Ogd & $a^{m+m!}b^{m}a^{m+m!}$ & ozn b.\\
  \hline 
    $\omega : |\omega|_{a} = |\omega|_{b} = |\omega|_{c}$ & LOP & $a^{n}b^{n}c^{n}$ & i=0\\
  \hline
    $\omega : |\omega|_{a} = |\omega|_{c} > |\omega|_{b}$ & LOP & $a^{n+1}b^{n}c^{n+1}$ &\\
  \hline
    $\omega\omega\omega$ & LOP & $0^{n}1^{n}0^{n}1^{n}0^{n}1^{n}$ & i=0\\
  \hline
    $\omega\omega^{R}\omega$ & LOP & $0^{n}1^{n}1^{n}0^{n}0^{n}1^{n}$ & i=0\\
  \hline
    $a^{n}c^{k}b^{n} : n \neq k$ & Ogd & $a^{n!+n}c^{n}b^{n!+n}$& \\
\end{tabular}

$ L = \{ w \in \{0, 1, 2 \}^* w = palindrom \; i \; |w|_0 = |w|_2 mod 13\} $ - hybryda palindromów (da sie zrobic automat) oraz przejście po stanach \\
L = $ \{ w \in \{a, b \}^* |w|_a = |w|_b + 5 \} $ < bezkontekstowy \\
L = $ \{ w \in \{a, b \}^* |w|_a = |w|_b = 2 mod 5 $ < regularny \\
\\

Majac $ L = \{ w \in \{a, b, c\}^* i |w|_a \neq |w|_b lub |w|_a \neq |w|_c lub |w|_b \neq |w|_c \} $ \\
mamy niedeterministyczny automat ze stosem ktory zgaduje poprawny wynik. \\
----------------------------\\

$L_{1} = \{w: w\in\{a,b,c\}^{*} \wedge |w|_{a} \neq |w|_{b} \neq |w|_{c}\}$
 
 Niech n stała z lematu Ogdena. Niech m > n. Wybieramy slowo $z = a^{m + m!}b^{m}c^{m + m!}$ i oznaczamy $m$ liter $b$ jako wyróżnione.
 \begin{enumerate}
 	\item nie możemy pompować samego $a$ ani samego $c$ (brak wyróżnionych).
 	\item nie możemy pompować jednocześnie $a$ oraz $c$ (pomiędzy nimi jest więcej niż $n$ wyróżnionych liter).
 \end{enumerate}
 
 
 Pozostają nam do rozpatrzenia podziały, w których:
 \begin{enumerate}
 	\item pompujemy b:\newline
 	wyznaczamy i: 
 	$|vx|_{b} = p$
 	\newline
 	$m+m! = m + (i-1)p$
 	\newline
 	$m! = ip - p$
 	\newline
 	$i = \frac{m!}{p} + 1$
 	\newline
 	\newline
 	$|z'|_{b}=|z|_{b} + (i-1)|vx|_{b} = m + (\frac{m!}{p} + 1 - 1)p = m + m!$
 	\newline
 	Długość b jest taka sama jak długość reszty więc wyszliśmy z języka. 
 	\item pompowanie a i b. Równamy ilość b do ilości c.
 	\newline
 	pomowanie b i c. Równamy ilość b do ilości a.
 \end{enumerate}
----------------------------\\
FIRST
\begin{enumerate}
  \item Szukamy produkcji gdzie na początku stoi terminal i ten terminal dodajemy do zbioru FIRST od nieterminala przed strzałką.
  \item Szukamy produkcji z eps i dodajemy ten eps do zbioru FIRST od nieterminala przed strzałką.
  \item Szukamy produkcji gdzie na początku stoi nieterminal i FIRST od tego nieterminala dodajemy do FIRST od nieterminala stojącego przed strzałką (bez epsilona). Jeżeli w kopiowanym zbiorze jest epsilon to dodajemy FIRST od następnego symbolu. (Jeśli w każdym symbolu jest epsilon to na końcu dodajemy epsilon).
\end{enumerate}

FOLLOW
\begin{enumerate}
  \item Do zbioru FOLLOW od symbolu początkowego dodajemy \$
  \item  Szukamy produkcji gdzie za nieterminalem będzie stał jakiś symbol i do FOLLOW od tego nieterminala dodajemy FIRST od następnego symbolu (pomijając eps). Jeśli w dodawanym zbiorze był eps to sprawdzamy kolejny symbol.
  \item Szukamy produkcji gdzie na końcu znajduje się nieterminal i do FOLLOW od tego nieterminala kopiujemy zawartość FOLLOW od nieterminala przed strzałką. 
  \item Szukamy produkcji gdzie za jakimś nieterminalem cała prawa strona będzie się zerowała (czyli w FIRST od całej strony będzie epsilon). Wtedy do FOLLOW od tego nieterminala dodajemy FOLLOW od nieterminala przed strzałką. \\
  Powtarzaj 3 i 4 dopóki są zmiany.
\end{enumerate}

LL
FIRST(alfa) - FIRST od pierwszego znaku \ {eps}, jeśli jest epsilon to wchodzimy do kolejnego

\\
bin(n)bin(n+1), n > 0 - nie jest bezkontekstowe \\
bin(n)bin(n+1)^R, n > 0 - bezkontekstowe \\
bin(n)#hex(n)^R, n > 0 > korzystamy z odpowiedniosci bin >> hex i ucieczka od zer na poczatku \\
